% Preview source code

%% LyX 2.1.4 created this file.  For more info, see http://www.lyx.org/.
%% Do not edit unless you really know what you are doing.
\documentclass[english]{article}
\usepackage[T1]{fontenc}
\usepackage[latin9]{inputenc}
\usepackage{amsmath}
\usepackage{graphicx}

\makeatletter

%%%%%%%%%%%%%%%%%%%%%%%%%%%%%% LyX specific LaTeX commands.
%% A simple dot to overcome graphicx limitations
\newcommand{\lyxdot}{.}


\makeatother

\usepackage{babel}
\begin{document}

\title{Revised Goodwin Model - Reconsidering ``Constants'' as Time Dependent
Variables}

\maketitle

\section*{The Goodwin Model}

We begin with the following assumptions for convince. The theoretical
economy being considered is a \emph{closed }economy, simply meaning
that there is no international trade.\\
\\
\textbf{(A1)} Technical Progress grows at a constant rate.\\
\textbf{(A2)} The labor force grows at a constant rate.\\
\textbf{(A3)} There exist only two homogeneous and non-specific factors
of production: capital and labor.\\
\textbf{(A4)} All quantities are real and net.\\
\textbf{(A5)} All wages are consumed; all profits are saved and invested.\\
\textbf{(A6)} There is a constant capital-output ratio.\\
\textbf{(A7)} A real wage that rises in the neighborhood of full employment,
expressed by the Phillips Curve.\\
\\
Further, the following list of abbreviations, definitions, and relations
describes the framework of the economy. 
\begin{itemize}
\item Output: $Y$
\item National Income: $q$
\item Labor: $L$ 
\item Capital: $k$
\item Wage Rate: $w$
\item Labor Productivity: $a$
\item Labor Income: $wL$
\item Labor Income Share: $u$
\item Profit Income Share: $\pi$
\item Savings: $(1-\frac{w}{a})Y$
\item Capital Output Ratio: $\sigma$
\item Labor Supply: $N$
\item Employment Rate: $v$
\end{itemize}
We begin with the tautological equation

\begin{equation}
q=\frac{Y}{L}\cdot\frac{L}{N}\cdot N
\end{equation}
 where $\frac{Y}{L}$ is defined to be labor productivity and $\frac{L}{N}$
is defined to be the employment rate. Thus the equation simplifies
to:
\begin{equation}
q=a\cdot v\cdot N.
\end{equation}
 We wish to take the logarithmic derivative with respect to time.
With regard to the construction of our model, taking the logarithmic
derivative allows us to understand how variables in our system change
over time. Then, taking the logarithmic derivative of (2) yields
\begin{equation}
\frac{\dot{q}}{q}=\frac{\dot{a}}{a}+\frac{\dot{v}}{v}+\frac{\dot{N}}{N}
\end{equation}
and simplifies to
\begin{equation}
\frac{\dot{q}}{q}=\alpha+\frac{\dot{v}}{v}+\beta.
\end{equation}
 Here, $\alpha$ is the increase in productivity - \emph{i.e. }$a=a_{0}e^{\alpha t}$
- and $\beta$ is the growth rate in labor supply - \emph{i.e. $N=N_{0}e^{\beta t}$}.
Recall from our assumptions, that both $\alpha$ and $\beta$ are
constants. 

The national income is distributed 100 percent between capitalist's
share of national income (\emph{$\pi$)} and worker's share of national
income ($v$), which can be mathematically expressed by $1=u+\pi$.
Further, by assumption, since all profits are reinvested, it follows
that 
\[
\dot{k}=q\cdot\pi
\]
\[
\frac{\dot{k}}{q}=1-u.
\]
 $\sigma$ represents a constant capital output ratio. Mathematically,
then, $\sigma=\frac{k}{q}$. Diving by $\sigma$ yields:
\begin{align*}
\frac{\dot{k}}{\sigma}= & \frac{(1-u)}{\sigma}\cdot q\\
\frac{q\cdot\dot{k}}{k}= & \frac{(1-u)}{k}q^{2}\\
\frac{\dot{k}}{k}= & \frac{(1-u)}{k}q\\
\frac{\dot{k}}{k}= & \frac{(1-u)}{\sigma}.
\end{align*}
We note some observations which will help us simplify our equation.
Note first that the equation above states that the growth rate of
capital is dependent on worker's share of national income. As a result,
we can replace the growth rate of capital with the growth rate of
national income. Similarly, since we know that $\sigma$ is constant,
the growth rate of national income must be the same as the growth
rate of the capital stock. Thus with these two assertions, a fluctuation
in capital leads directly to a fluctuation in the national income:
\begin{equation}
\frac{\dot{k}}{k}=\frac{\dot{q}}{q}=\frac{1-u}{\sigma}.
\end{equation}
Setting (4) equal to (5) and solving for $\dot{v}$, we get the first
equation differential equation describing employment rate:
\begin{equation}
\dot{v}=v(t)\left(\frac{1}{\sigma}-(\alpha+\beta)-\frac{u(t)}{\sigma}\right).
\end{equation}


The second equation is established in a similar manner. To assist
in defining it, we begin with the Phillips Curve. Phillips \cite{Phillips58}
attempted to estimated a correlation between changes in wages and
unemployment rate. Goodwin took advantage of this relationship, however
for simplicity, he proposed a linearized version. While the exact
relationship is not known, this linearized version will suffice to
model the relationship. To transform the unemployment rate into the
employment rate, we begin by noting that the labor supply in our economy
is comprised fully of employed and unemployed workers, \emph{i.e.}
$1=v+z$ where $z$ is defined to be the unemployment rate. Figure
1 graphically notes the transformation of the Phillips curve from
an unemployment rate to an employment rate. 

%\begin{figure}
%\centering{}\includegraphics[scale=0.5]{\string"Thesis Figures/Screen Shot 2016-12-01 at 2.12.09 PM\string".png}\caption{Transformation of the Phillips Curve}
%\end{figure}


With the transformed and linearized Phillips curve, we note
\begin{equation}
\frac{\dot{w}}{w}=-\gamma+\rho v.
\end{equation}
Here, $\gamma$ is defined to be the lines intersection with the $y$-axis
and $\rho$ is the slope of the line. We define worker's share of
national income as 
\begin{equation}
u=\frac{w}{a}.
\end{equation}
A logarithmic differentiation with respect to time yields the growth
rate of worker's share of national income
\[
\frac{\dot{u}}{u}=\frac{\dot{w}}{w}-\frac{\dot{a}}{a}
\]
\begin{equation}
\frac{\dot{u}}{u}=\frac{\dot{w}}{w}-\alpha.
\end{equation}
 Finally, plugging (7) into (9) and solving for $u$ yields the following
equation for change in worker's share of national income:
\begin{equation}
\dot{u}=u(t)\left(-(\alpha+\gamma)+(\rho v(t))\right)
\end{equation}
The differential equation describing the dynamics of worker's share
of national income is analogous to the equation describing the predator
population in the Lotka-Volterra model.

To summarize, we have a system of differential equations to describe
the Goodwin Model:

\begin{align}
\dot{v}= & v(t)\left(\frac{1}{\sigma}-(\alpha+\beta)-\frac{u(t)}{\sigma}\right)\label{eq:2.17-1}\\
\dot{u}= & u(t)\left(-(\alpha+\gamma)+(\rho v(t))\right).\nonumber 
\end{align}



\section*{Revised Goodwin Model}

While the original Goodwin model is important and valuable - the simplicity
with which it is written and the dynamic properties it yields are
both interesting from a mathematical and economic perspective - it
is by no means a realistic model of a countries economy. Goodwin himself
commented that his model is ``quite unrealistic . . . of cycles in
growth rates'' for a given country. Despite its limitations, the
model has proved to be remarkably popular and has inspired a huge
literature. Part of the model's enduring appeal lies in its simplicity;
in particular, the elegance with which it illustrates the cyclical
relationship between income distribution and employment in a dynamic
capitalist economy over the course of the business cycle. Because
of this, many revisions to Goodwin's original model have been proposed,
all which look to more accurately model cycles of growth rates in
an economy. 

This section looks to do just that. When considering the constants
outlined in Goodwin's model, it is not unrealistic to expect some
variability in them, especially over great periods of time. This becomes
even more apparent when trying to find data to ultimately test Goodwin's
proposed model. The most straightforward way to estimate Goodwin's
constants is through a time series dataset. There are two possibilities
for the time series used to estimate the constants, each of which
can potentially cause problems with the dynamics of the theoretical
model. An estimated time series can either be stationary or unstationary
process. If a time series is found to be unstationary, the problem
is obvious: clearly, the estimation for a given ``constant'' is
far from constant and would therefore be inappropriate to simply take
an average. However, a stationary series still presents its own problems;
the variance of the time series could be so large that, while the
series is stationary, values are varying so greatly that simply taking
an average of the time series would not work. It seems likely, therefore,
that when testing Goodwin's model, instead of estimating constants,
we will have variables in our system in place of our constants. It
needs to be determined, therefore, if looking at these constants as
time series effects the structure, and ultimately the dynamics, of
the model. 

There are five constants present in Goodwin's model: $\alpha$, $\beta$,
$\sigma$, $\rho$, and $\gamma$. Both $\rho$and $\gamma$are estimated
through a simple linear regression model; they will remain constants
in our model. The main concern is in the estimation of $\alpha$,
$\beta$, and $\sigma$. If any of these constants are derived through
some process of taking derivatives, then our model will change. As
we take a derivative of a time dependent variable, rather than a constant,
additional time dependent terms will pop out of the derivation. Recall
that both $\alpha$ and $\beta$ are derived through taking the logarithmic
derivative of (2). We therefore suspect that when considering these
values as time series, rather than as constants, the structure of
the Goodwin model will change. 

Notice that equation (1) can be rewritten as 
\[
q=a\cdot l.
\]
 In Goodwin's original derivation of the model, he looks at the logarithmic
derivative of $a$, which can be rewritten as $\frac{q}{l}$. Therefore,
$\dot{a}=\frac{d}{dt}\left(\frac{q}{l}\right)$. Taking the logarithmic
derivative of $a$, we are left with a new expression:
\begin{eqnarray*}
\frac{\dot{a}}{a} & = & \frac{\frac{d}{dt}(a_{0}e^{\alpha t})}{a}\\
 & = & \frac{a_{0}e^{\alpha t}(\dot{\alpha}t+t)}{a}\\
 & = & \frac{a\cdot(\dot{\alpha}t+t)}{a}\\
 & = & \dot{\alpha}t+t
\end{eqnarray*}
since $\alpha$ is now a function of time. Similarly, notice that
\begin{eqnarray*}
\frac{\frac{d}{dt}\left(\frac{q}{l}\right)}{\left(\frac{q}{l}\right)} & = & \frac{l\dot{q}+\dot{l}q}{l^{2}}\cdot\frac{l}{q}\\
 & = & \frac{\dot{q}}{q}-\frac{\dot{l}}{l}.
\end{eqnarray*}
We are left with 
\[
\frac{\dot{q}}{q}-\frac{\dot{l}}{l}=\dot{\alpha}t+t.
\]
We can see that we can rewrite this expression in terms of $\dot{l}$,
which will be of some assistance later in our derivation of the model:
\begin{eqnarray*}
\frac{\dot{l}}{l} & = & \frac{\dot{q}}{q}-(\dot{\alpha}t+t)\\
 & = & \left(\frac{1-u}{\sigma}\right)-(\dot{\alpha}t+t)
\end{eqnarray*}
where $\frac{\dot{q}}{q}=\left(\frac{1-u}{\sigma}\right)$, which
is shown in the original derivation of the model.

Similarly, observe that when consider the logarithmic derivative of
$v$, we will get a new expression in place of $\beta$. Since $v=\frac{l}{N}$,
we have that $\dot{v}=\frac{d}{dt}\left(\frac{l}{N}\right)$. Thus,
\begin{eqnarray*}
\frac{\dot{v}}{v} & = & \frac{\frac{d}{dt}\left(\frac{l}{N}\right)}{\frac{l}{N}}\\
 & = & \frac{N\dot{l}+\dot{N}l}{N^{2}}\cdot\frac{N}{l}\\
 & = & \frac{\dot{l}}{l}-\frac{\dot{N}}{N}
\end{eqnarray*}
To simplify, we need to find an expression for $\frac{\dot{N}}{N}$,
since we already have an expression for $\frac{\dot{l}}{l}$. Notice
that 
\begin{eqnarray*}
\frac{\dot{N}}{N} & = & \frac{\frac{d}{dt}(N_{0}e^{\beta t})}{N}\\
 & = & \frac{N\cdot(\dot{\beta}t+t)}{N}\\
 & = & \dot{\beta}t+t
\end{eqnarray*}
which is different from our original expression of just $\beta$ because
$\beta$ is now a time dependent variable. Substituting our expression
of $\frac{\dot{l}}{l}$ and $\frac{\dot{N}}{N}$ and solving for $\dot{v}$,
we are left with the first revised equation in the Goodwin model:
\[
\dot{v}=v(t)\left(\frac{1-u(t)}{\sigma}-\left[(\dot{\alpha}t+t)+(\dot{\beta}t+t)\right]\right).
\]


The second equation in the Goodwin model can be rederived in a similar
fashion, whereby treating $\alpha$ as a time dependent derivative.
Recall that $u=\frac{w}{a}.$ Then, taking the logarithmic derivative
\begin{eqnarray*}
\frac{\dot{u}}{u} & = & \frac{\dot{w}}{w}-\frac{\dot{a}}{a}\\
 & = & (-\gamma+\rho v)-(\dot{\alpha}t+t)
\end{eqnarray*}
where $\frac{\dot{w}}{w}$ is defined in the original derivation of
the model. Hence, solving for $\dot{u}$, we are left with the second
equation in the revised Goodwin Model:
\[
\dot{u}=u(t)\left(\gamma+\rho v(t)-(\dot{\alpha}t+t)\right)
\]
Our model is therefore summarized by the following two equations:
\begin{eqnarray*}
\dot{v} & = & v(t)\left(\frac{1}{\sigma}-\left[(\dot{\alpha}t+t)+(\dot{\beta}t+t)\right]-\frac{u(t)}{\sigma}\right)\\
\dot{u} & = & u(t)\left(\gamma+\rho v(t)-(\dot{\alpha}t+t)\right).
\end{eqnarray*}
 
\end{document}
